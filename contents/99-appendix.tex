\section*{Appendix}
% Ein Anhang zur wissenschaftlichen Arbeit ist notwendig, wenn Materialien, die die Arbeit als Ganzes oder auch größere Teile derselben betreffen, jedoch nur schwer im Ausführungsteil unterzubringen sind. Das ist insbesondere dann der Fall, wenn sie aufgrund ihres Umfangs den Gesamtzusammenhang der Ausführung stören würden. Inhaltlich darf im Anhang nichts stehen, was zum Verständnis des Textes notwendig ist, der Text der Arbeit darf an dieser Stelle nicht „unter anderen Vorzeichen“ fortgesetzt werden. Er sollte nicht dazu verwendet werden, der Arbeit einen größeren Umfang zu geben und diese „dicker“ erscheinen zu lassen!

% Der Anhang eignet sich für ergänzende Dokumente und Materialien, vor allem, falls diese für den Leser nur schwer oder gar nicht zugänglich sind, wie bspw. unveröffentlichte Betriebsunterlagen.
% Vor allem in den empirischen Arbeiten kann der Anhang dazu dienen, verwendete Datensätze, eingesetzte mathematisch-statistische Verfahren oder Programme näher zu kennzeichnen. Werden im Rahmen der Untersuchungen Befragungen durchgeführt, sind die Fragestellungen und Ergebnisse im Anhang zu dokumentieren. Auf Gespräche darf im Rahmen der Ausführungen nur dann Bezug genommen werden, wenn ein vom Gesprächspartner unterzeichnetes Ergebnis-Protokoll im Anhang der Arbeit beigefügt ist.

% Besteht der Anhang aus mehreren Elementen, so sind die einzelnen Elemente durch Nummerierung voneinander zu trennen.

% ignore spacing issues in the appendix, since tables tend to underflow
\hbadness 10000

\subsection*{System Prompts}

\lstinputlisting[caption={System Prompt with \ac{CoT} Explanations}]{code/system_prompt-cot.txt}

\clearpage

\lstinputlisting[caption={System Prompt without Explanations}]{code/system_prompt-plain.txt}

% LTeX: enabled=false
\newpage
\subsection*{Questionnaires}

\subsubsection*{Technology Acceptance Model}

\begin{ctable}
    \begin{tabularx}{\textwidth}{l|X|X}
        \textbf{ID} & \textbf{English} & \textbf{German} \\
        \hline
        01 & Using CHART-MASTER in my job would enable me to accomplish tasks more quickly. & Die Nutzung des Chatbots würde mir helfen die Matheaufgaben schneller zu lösen. \\
        02 & Using CHART-MASTER would improve my job performance. & Die Nutzung des Chatbots würde mir helfen bessere Noten in Mathe zu erzielen. \\
        03 & Using CHART-MASTER in my Job would increase my productivity. & Die Nutzung des Chatbots würde mir helfen bessere Lernerfolge zu erzielen \\
        04 & Using CHART-MASTER would enhance my effectiveness on the job. & Die Nutzung des Chatbots würde mir helfen effektiver zu lernen. \\
        05 & Using CHART-MASTER would make it easier to do my job. & Die Nutzung des Chatbots würde das Lernen einfacher machen. \\
        06 & I would find CHART-MASTER useful in my job. & Ich finde den Chatbot hilfreich beim lernen. \\
    \end{tabularx}
\end{ctable}

\begin{ctable}
    \begin{tabularx}{\textwidth}{l|X|X}
        \textbf{ID} & \textbf{English} & \textbf{German} \\
        \hline
        01 & Learning to operate CHART-MASTER would be easy for me. & Den Umgang mit dem Chatbot zu lernen würde mir leicht fallen. \\
        02 & I would find it easy to get CHART-MASTER to do what I want. & Ich fände es einfach, den Chatbot dazu zu bringen zu tun, was ich von ihm möchte. \\
        03 & My interaction with CHART-MASTER would be clear and understandable. & Meine Interaktion mit dem Chatbot wäre einfach zu verstehen. \\
        04 & I would find CHART-MASTER to be flexible to interact with. & Ich fände den Chatbot flexibel in seinen Interaktionsmöglichkeiten. \\
        05 & It would be easy for me to become skillful at using CHART-MASTER. & Es wäre einfach guten Umgang mit dem Chatbot zu erlernen. \\
        06 & I would find CHART-MASTER easy to use. & Ich fände den Chatbot einfach zu nutzen. \\
    \end{tabularx}
\end{ctable}

\clearpage
\subsubsection*{Computer Self-Efficacy}

\begin{ctable}
    \begin{tabularx}{\textwidth}{l|X|X}
        \textbf{ID} & \textbf{English} & \textbf{German} \\
        \hline
        00 & I could complete the job using the software package... & Ich könnte die Matheaufgaben mithilfe des Chatbots lösen wenn... \\
        \hline
        01 & ...if there was no one around to tell me what to do as I go. & ...niemand mich anleitet, während ich die Aufgaben löse. \\
        02 & ...if I had never used a tool like it before. & ...ich noch nie ein LLM genutzt hätte. \\
        03 & ...if I had only the software manuals for reference. & ...ich nur eine Anleitung für den Chatbot als Hilfe hätte. \\
        04 & ...if I had seen someone else using it before trying it myself. & ...ich vorher gesehen hätte wie jemand mit dem Chatbot Matheaufgaben löst. \\
        05 & ...if I could call someone for help if I got stuck. & ...ich jemanden um Hilfe fragen könnte, falls ich Probleme habe. \\
        06 & ...if someone else had helped me get started. & ...mir jemand anfänglich hilft. \\
        07 & ...if I had a lot of time to solve the problems for which the tool was provided. & ...ich viel Zeit hätte die Matheaufgaben zu lösen. \\
        08 & ...if I had just the built-in help facility for assistance. & ...ich nur die im Chatbot verbauten Hilfen zur Verfügung hätte. \\
        09 & ...if someone showed me how to do it first. & ...jemand mir zuerst zeigt wie man den Chatbot nutzt. \\
        10 & ...if I had used a similar tool before this one to solve the same problems. & ...ich vergleichbare Chatbots bereits zum Lösen von Matheaufgaben genutzt hätte. \\
    \end{tabularx}
\end{ctable}

\clearpage
\subsubsection*{NASA-TLX}

\begin{ctable}
    \begin{tabularx}{\textwidth}{l|X|X}
        \textbf{ID} & \textbf{English} & \textbf{German} \\
        \hline
        01 & How mentally demanding was the task? & Wie hoch war die mentale Anforderung beim Lösen der Aufgaben? \\
        02 & How physically demanding was the task? & Wie hoch war die körperliche Anforderung beim Lösen der Aufgaben? \\
        03 & How hurried or rushed was the pace of the task? & Wie hoch war der Zeitdruck beim Lösen der Aufgaben? \\
        04 & How successful were you in accomplishing what you were asked to do? & Wie hoch ist Ihrer Zufriedenheit mit Ihrer Leistung beim Lösen der Aufgaben? \\
        05 & How hard did you have to work to accomplish your level of performance? & Wie hoch war die empfundene Anstrengung beim Lösen der Aufgaben? \\
        06 & How insecure, discouraged, irritated, stressed, and annoyed were you? & Wie frustriert, genervt, gestresst oder entmutigt haben Sie sich beim Lösen der Aufgaben gefühlt? \\
    \end{tabularx}
\end{ctable}

% LTeX: enabled=true
