\section{Einleitung} \label{einleitung}

Dieser Teil der Arbeit sollte folgende Inhalte haben:

\begin{itemize}
\item  Einführung in die Situation und Problemstellung
\item  Lösung und eigener Beitrag
\item  Vorgehen und Aufbau der Arbeit
\end{itemize}

Grundsätzlich sollten Kapitelüberschriften sprechend sein, das gilt insbesondere für das Einleitungskapitel, denn grundsätzlich ist das erste Kapitel immer ein einleitendes Kapitel. Der Leser würde also mit der Überschrift ``Einleitung'' nichts über den Inhalt Ihres Kapitels erfahren. Eine Untergliederung der Einleitung ist nicht üblich.

Hinweis:
Es hat sich als hilfreich erwiesen, die Einleitung mit der Zusammenfassung bzw. dem Abstract und der Schlussfolgerung zu vergleichen. Damit stellt man sicher, dass diese inhaltlich im Bezug auf Zielsetzung und Motivation übereinstimmen. Der Umfang sollte ca. 5\% der gesamten Arbeit betragen.
