
\section{Zusammenfassung} \label{zusammenfassung}

In der Zusammenfassung sollen
\begin{itemize}
\item die Themenstellung
\item der gewählte Ansatz
\item die Ergebnisse der Arbeit
\item eine kritische Stellungnahme/Einschätzung sowie die Limitationen Ihrer Forschung
\item nächste Schritte
\end{itemize}
deutlich werden.
Es ist keine Nacherzählung dessen, was sie chronologisch aufgeschrieben haben.

\textbf{Hinweis:}
Die Schlussfolgerung sollte mit der Zusammenfassung bzw. dem Abstract und der Einleitung abgeglichen werden. Es sollte immer eine Zusammenfassung der wesentlichen Erkenntnisse der eigenen Arbeit sein, die den Forschungsbeitrag darstellt. Der Umfang der Schlussfolgerung sollte ähnlich wie die Einleitung ca. 5\% der gesamten Arbeit betragen.


Bitte Rechtschreibprüfung nicht vergessen.